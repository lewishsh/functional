\documentclass[12pt]{article}
\newcommand{\R}{\mathbb{R}}
\newcommand{\ran}{\text{ran}}
\renewcommand{\H}{\mathcal{H}}
\newcommand{\N}{\mathcal{N}}
\usepackage{microtype}
\usepackage{enumitem}
\usepackage{amsmath}
\usepackage{amssymb}
\usepackage[margin=1.25in]{geometry}
\usepackage{enumitem}
\begin{document}
\begin{center}
  Lewis Ho\\
  Functional Analysis\\
  Problem Set 4
\end{center}

\paragraph{Problem 1}

$\overline{A(B_1(0))}$ must contain a ball, else the range of $A$ will be
$\bigcap_{n=1}^\infty A(B_n(0)) = \bigcap_{n=1}^\infty nA(B_1(0))$ which will
then be meagre. Because $\overline{A(B_1(0))}$ is symmetric and convex, it will
in particular contain a ball $B_{\varepsilon_0/2}(0)$, as if the original ball is
centered around some $y_0$, then $(-y_0 + y_0 + B_{\varepsilon_0}(0))/2$ will be
a set of convex combinations of points we know to be in the closure.

Then we show $\overline{A(B_1(0))}\subset A(B_2(0))$, following the proof of the
open mapping thoerem. Suppose $y \in \overline{A(B_1(0))}$, i.e. $\exists x_1\in
B_1(0)$ such that $\|y-Ax_1\| < \varepsilon/4$, i.e. $y-Ax_1 \in B_{\varepsilon_0
  /4}\in\overline{A(B_{1/2}(0)}$. Because it's contained in this ball, there exists
some $x_2 \in B_{1/2}(0)$ such that $\|y-Ax_1-Ax_2\|<\varepsilon_0/8$, i.e.
$y - Ax_1 - Ax_2 \in B_ {\varepsilon_0/8}(0) \subset \overline{A(B_{1/4})}$.
Generally, there exists $x_n \in B_{1/2^{n-1}}(0)$ such that $\|y-\sum_{i=1}^nx_i\|
< \varepsilon_0/2^{n+1}$.

We know that in Banach spaces if $\sum\|x_n\|<\infty$ then $\sum x_n$ converges,
and $\sum\|x_n\|$ in this case $ \leq \sum_{n=1}^\infty \frac{1}{2^{n-1}} \leq 2$.
Thus the limit $x$ of this sum exists, and by the triangle inequality, is in
$B_2(0)$, and finally satisfies $Ax = y$. Hence $y \in A(B_2(0)$. From this we
conclude (by scaling) that $A(B_1(0))$ contains a ball of $\varepsilon_0/4$ radius,
and thus by multiplying $B_1(0)$ by arbitrarily large numbers, can conclude that
$A$ is surjective and thus has a closed range.

\paragraph{Problem 2}

The statement is false. We can construct arbitrarily tall isosceles triangles of
$\varepsilon$ area. Center those triangles at 0.5, and define functions $f_
h$ to be the functions that are 0 outside the base of the triangle of height $h$,
and take the value of the diagonals up and down inside. $\|f_h\|$ is
unbounded, but there is always a function $F_h$ whose supremum (and thus norm) is
$\varepsilon$ and has derivative $f_h$, namely the antiderivative of $f_h$.

\paragraph{Problem 3}

If there exists some $c$ such that $\|Ax\|\geq c\|x\|$ for all $x$, then $\N(A)$
must empty, otherwise arbitrarily large vectors in the null space can have images
with norm zero, a contradiction. Closedness: let $Af_n$ be a Cauchy sequence in
the range. For any $\|Af_n-Af_m\| < \varepsilon$, by the assumption,
\begin{displaymath}
  \|Af_n-Af_m\| = \|A(f_n-f_m)\| \geq c\|f_m-f_n\|,
\end{displaymath}
in other words, $f_n$ must also be Cauchy, converging to some $f$ in the domain.
Then by continuity, $Af = A\lim_{n\to\infty}f_n = \lim_{n\to\infty}Af_n$, i.e. the
range is closed.

Conversely, if both $\N(A) = \{0\}$ and the range is closed, then if there were
to not exist a $c$ such that $\|Ax\|\geq c\|x\|$, we could construct a
sequence $Af_n\to 0$, where $\|f_n\| = 1$. Because the range is closed, there is
some $f$ with norm 1 such that $Af = 0$, violating our $\N(A) = \{0\}$
assumption.

\paragraph{Problem 4}
By a modification of H\"older's inequality, we get: $(x_n,y) \leq \|x_n\|_p
\|y\|_q$.

Suppose our assumptions hold $\sup_n\|x_n\|_p<\infty$ and $x_n^j\to 0$. I will
show that we can make $(x_n,y)$ arbitrarily small. Firstly, note that we can
bound $(x_n,y)$ by decomposing it and applying ``H\"older's'' as follows:
\begin{displaymath}
  |(x_n,y)| =
  |\sum_{|y^j|>\delta}x_n^jy^j + \sum_{|y^j|\leq \delta}x_n^jy^j|
  \leq \|y\|_q\left(\sum_{|y^j|>\delta}|x_n^j|^p\right)^{\frac{1}{p}} +
  \left(\sum_{|y^j<\delta|}|y^j|^q\right)^{\frac{1}{q}}\sup_k\|x_k\|_p
\end{displaymath}
Note that we can always find an $N$ such that for all $n > N$ both parts are
arbitrarily small. First we can fix $\delta$ to make the latter term small,
e.g. because $\|y\|_q$ is finite, we can look at the first $N$ terms such that
their $q$ sum is arbitarily close to $\|y\|^q_q$, and then set $\delta$ to be
their infimum. With this fixed $\delta$, there are now finite elements in
the sum of $x^j_n$s in the first term, and thus we can pick some $N$ such that
for $n>N$ all $x^j_n$ are arbitrarily small, and thus their finite $p$ sum is
also arbitrarily small.

For the converse, note that $x_n$ define bounded linear operators from $\ell^q$ to
$\mathbb{R}$. If $(x_n,y)\to 0$ for all $y$, then the set $E = \{y\in\ell^q\ |
\ \sup_n\|(x_n,y)\| < \infty\} = \ell^q$, and by the uniform boundedness principle
$\sup_n\|x_n\|_{p}$ must be bounded. Finally,
if $x_n^j \to 0$ is violated, the sequence with $j$th element = 1 has image not 
converging to 0.

\paragraph{Problem 5}

In the previous problem set we showed that ${c_0}^* = \ell^1$. NEED TO SHOW
THE NORM OF ELL IS THE NORM OF THE SPACE.

To show $(x_n,y)\to 0$, we decompose $(x_n,y)$ again as follows:
\begin{displaymath}
  |(x_n,y)| =
  |\sum_{|y^j|>\delta}x_n^jy^j + \sum_{|y^j|\leq \delta}x_n^jy^j|
  \leq \|y\|_\infty\sum_{|y^j|>\delta}|x_n^j| + \delta \sup_n\|x_n\|_1
\end{displaymath}
As previously, we can choose $\delta$ such that the second term becomes
arbitrarily small, and then given this delta, choose $n$ such that the first
term becomes arbitrarily small.

To show the converse, note that $x_n$ define linear operators from $c_0 \to
\mathbb{R}$. If $(x_n,y)\to 0$ for all $y$, then the set $E$ for $\{x_n\}$ is
all of $c_0$, and thus $sup_n\|x_n\|$ is finite. Additionally if we have some
$x_n^j$ not converging to $0$, then $e_j$ has image not converging to zero.

\paragraph{Problem 6}

Let $h = \sum_ia^ie_i$, and $h_n = \sum_ih_n^ie_i$. If our assumptions are
satisfied, we can write $|(h_n,h)|$ as
\begin{displaymath}
  |\sum_{|a^i|>\delta}a^ih_n^i + \sum_{|a^i|>\delta}a^ih_n^i| \leq
  \|h\|\left(\sum_{|a^i|>\delta}|h_n^i|^2\right)^{\frac{1}{2}} +
  \left(\sum_{|a^i|<\delta}|a^i|^2\right)^{\frac{1}{2}}\sup_k\|h_k\|
\end{displaymath}
using the Cauchy-Schwarz inequality to bound it. Then as before, the latter
term is made arbitrarily small with our choice of $\delta$, and the former
with a sufficiently large choice of $N$.

For the converse, we use the uniform boundedness principle, as all $h_n$ define
operators $\mathcal{H}\to \mathbb{R}$, and $E$ is all of space for $\{h_n\}$.
Likewise if some $b^j_n$ doesn't go to zero, $(e_j,h)$ won't converge to zero
either.

\paragraph{Problem 7}

Proof by contradiction: suppose this statement is false, i.e. there exists
some $\varepsilon_0$ such that for all $N$ there exists some $n>N$ such that
$\sup_f\|A_nf-Af\| > \varepsilon_0$. From this we can construct a sequence
$\{f_n\}$ where $\|A_nf_n - Af_n\|> \varepsilon_0$, and then find the limit
$f$ of a convergent subsequence $\{f_{n_k}\}$ using compactness.

Consider $\|A_nf-Af\|$. Using our subsequence $\{f_{n_k}\}$ and their
accompanying $A_{n_k}$, we can write
\begin{align*}
  \|A_{n_k}f - Af\|
  &= \|(A_{n_k}-A)(f_{n_k} + f - f_{n_k})\|\\
  &= \|A_{n_k}f_{n_k}-Af_{n_k} + (A_{n_k}-A)(f-f_{n_k})\|\\
  &\geq \varepsilon_0 - (\sup_n\|A_{n}\|+\|A\|)\|f-f_{n_k}\|
\end{align*}
and that we can choose large enough $n_k$ such $f$ and $f_{n_k}$ are very close
and this difference is bounded from
below by, say, $\varepsilon_0/2$, meaning that it doesn't converge. Note that
$\sup_n\|A_n\|$ and $\|A\|$ exist by the uniform boundedness principle.

\paragraph{Problem 8}

Let $M = \sup_{A \in \mathcal{A}}\|A\|$, which exists because the image set of the
unit ball is compact (and hence bounded), and thus by the uniform boundedness
principle the supremum of their norms is finite. From the previous problem and
both the pointwise convergence of $L_n$ and the collective compactness of
$\mathcal{A}$, we know that:
\begin{align*}
  & \sup_{g\in \overline{\mathcal{A}(B_1(0))}}\|L_ng-Lg\| \to 0\\
  \Rightarrow &\sup_{\|f\| \leq 1,\ A\in \mathcal{A}}\|(L_n-L)Af\| \to 0
\end{align*}
the first implying the second because:
\begin{displaymath}
  \overline{\mathcal{A}(B_1(0))} \subseteq \mathcal{A}(B_1(0)) =
  \{Af\ |\ A\in \mathcal{A},\ f\in B_1(0)\}.
\end{displaymath}
Now we invoke the definition of the norm of an operator $T$ as $\sup\|Tf\|$
for $\|f\|\leq 1$ to infer
\begin{displaymath}
  \sup_{\|f\|\leq 1,\ A\in \mathcal{A}}\|(L_n-L)Af\| = \sup_{A\in\mathcal{A}}
  (\sup_{\|f\|\leq 1}\|(L_n-L)Af\|) = \sup_{A\in\mathcal{A}}\|(L_n-L)A\|\to 0.
\end{displaymath}

\paragraph{Problem 9}

We first show that for sufficiently large $n$, $I-A_n$ is injective. Because
a subset of a compact set is itself precompact, $A_n$ are all compact
operators. Because of this, 




\end{document}