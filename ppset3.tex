\documentclass[12pt]{article}
\renewcommand{\chi}{\mathcal{X}}
\renewcommand{\null}{\mathcal{N}}
\newcommand{\ran}{\mathcal{R}}
\newcommand{\R}{\mathcal{R}}
\renewcommand{\H}{\mathcal{H}}
\newcommand{\K}{\mathcal{K}}
\newcommand{\intr}{\int_\mathbb{R}}
\usepackage{microtype}
\usepackage{enumitem}
\usepackage{amsmath}
\usepackage{amssymb}
\usepackage[margin=1.25in]{geometry}
\usepackage{enumitem}

\begin{document}
\begin{center}
  Lewis Ho\\
  Functional Analysis\\
  Practice Pset 3
\end{center}

\paragraph{Problem 1}
$x_0$ is an extreme point of $F$, because there are no other points in $F$ and
so no points of which $x_0$ could be a convex combination of. Then recall
the theorem that $\mathcal{E}(F) = F \cap \mathcal{E}{A}$ if $F$ is an extreme
set of $A$, closed, convex and compact. Thus $x_0 \in \mathcal{E}(F) \subseteq
\mathcal{E}(A)$.

\paragraph{Problem 2}
The open mapping theorem implies $T(B_1(0)$ contains a ball of radius
$\varepsilon$. Thus $T^{-1}(x) \in B_1(0)$ for $x \in B_\varepsilon(0)$, and
$\|T\| \leq 1/\varepsilon$.

\paragraph{Problem 3}
Suppose $T$ is an open map. $B_1(0)$ being an open set, $T(B_1(0)$ must be
open as well. $T(0) = 0 \in T(B_1(0)$, thus there must be some $B_\varepsilon(0)
\subseteq T(B_1(0))$. Conversely, let $A$ be an open set. For any $x$ in $A$,
there is some $B_\varepsilon(x)\subseteq A$. If $T(B_1(0))$ contains an open ball
of radius $\delta$,
\begin{displaymath}
  T(B_\varepsilon(x)) = x + \varepsilon T(B_1(0))
\end{displaymath}
contains a ball of radius $\epsilon\delta$, which is a subset of $T(B_\varepsilon
(x))\subseteq T(A)$, and thus $T$ is an open map.

\paragraph{Problem 4}
We can see that $F$ is nonempty because if we construct a sequence $x_n$ where
$\ell(x_n) - \sup\ell(y) \leq 1/n$, by compactness we have a convergent
subsequence whose limit, $x$, must satisfy $\ell(x) = \sup\ell(y)$. And because
$B$ is closed, $x\in B$ and $F$, so $F$ is nonempty.

Closedness follows from the continuity of $\ell$ and the closedness of $B$---
specifically, if $a_n\in F$, then their limit $a$ must satisfy $\ell(a) =
\sup\ell(x)$ by continuity, and must be in $B$. Convexity follows from the
linearity of $\ell$.

Extremity: suppose for some $x \in F$ there exists $a,\ b$ in $A$ such that
$x$ is a convex combination of $a$ and $b$. Then $a$ and $b$ must be in $F$
because if, say, $a$ wasn't, then $\ell(b)$ must be greater than $\ell(x)$,
and that's not possible as $x$ is the supremum of $\ell$ over $B$, and $a$ and
$b$ must be in $B$ because $B$ is an extreme subset.

\paragraph{Problem 5}
We show that $f$ in the ball can always be written as a convex combination of
two other functions in the ball. If $f = 0$, then any $g$ and $-g$ in the ball
will suffice. If $f \neq 0$, then $f$'s support has nonzero measure. We can
split that support in two and have $g = f$ on one part of that and zero elsewhere
and $h$ defined likewise for the other half. Then $g + h = f$, and clearly
both must have norm less than one.

\paragraph{Problem 6}
Our distance function satisfies non-negativity, identity of indiscernibles,
symmetry, and subadditivity, so it defines a metric space.

Consider the set defined by $a_i \leq 1/i$. Compactness
follows from the fact that this is the range of the closure of finite-rank
operators. On the other hand, $\sum$


\end{document}