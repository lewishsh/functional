\documentclass[12pt]{article}
\newcommand{\R}{\mathbb{R}}
\renewcommand{\H}{\mathcal{H}}
\newcommand{\s}{\sigma}
\renewcommand{\l}{\lambda}
\newcommand{\C}{\mathbb{C}}
\usepackage{microtype}
\usepackage{amsmath}
\usepackage{amssymb}
\usepackage[margin=1.25in]{geometry}
\usepackage{enumitem}
\begin{document}
\begin{center}
  Lewis Ho\\
  Functional Analysis\\
  Problem Set 7\\
  Collaborator: Anton
\end{center}

\paragraph{Problem 1}
Lemma: if $\l \in \s_r(T)$, $\bar{\l} \in \s_p(T^*)$. Proof: if the range of
$T - \l I$ is not dense, by Hahn Banach there exists a linear functional
vanishing on the entirety of the range. Then, for all $v$,
\begin{displaymath}
  0 = (v^*,(T-\l I)v) = ((T^*-\bar{\l}I)v^*, v),
\end{displaymath}
and thus $\l$ is an eigenvalue for $T^*$.


\subparagraph{$\s_p(R)$, $\s_r(L) = \O$:}
For all $\l$, $R-\l I$ has a trivial kernel.
Proof: if $\l$ is zero, then it has a trivial kernel. For $\l \neq 0$, 
let $v$ be in the kernel of $R-\l I$. Then as $Rv = \l v$, and as $(Rv)_1 =
0$ by the nature of the right-shift operation, $v_1 = 0$. Additionally, if
$v_n = 0$, $(Rv)_{n+1} = \lambda v_n = 0$, and hence $v_{n+1} = 0$. By induction,
$v_n = 0$ for all $n$. This shows that there are no eigenvectors for any value
of $\l$, and thus $\s_p(R) = \s_r(L) = \O$. Finally, by our lemma, $\s_r(L)
= \O$.

\subparagraph{$\s_p(L)$, $\s_r(R) = \{\l\in\C: |\l |<1\}$:}
Consider the vector $(1, \l, \l^2, \l^3, \ldots)$. $L - \l I$ applied to this
vector gives 0. Further, the norm of this vector is a geometric series with
ratio $\l^2$, and thus has a finite sum when $|\l|<1$. Hence our operator has
a nontrivial kernel and $\{\l\in\C: |\l |<1\} \subset \s_p(L)$. To show equality,
note that $|\l|> 1$ yields a set in the resolvent set. Thus we consider $|\l| = 1
$. Suppose then that $L-\l I$ has a nontrivial kernel. Then $v$ in the kernel
has nonzero element $v_k$. As $Lv = \l v$, $v_{k+1} = \l v_k$. And by induction
all subsequent elements $v_n$ have $|v_n| = |v_k|$, which is not a square-summable
sequence, a contradiction.

By question 4, combined with the emptiness of $\s_p(R)$ imply $\s_p(L) = \s_r(R)$.



\subparagraph{$\rho(L), \rho(R) = \{\l\in\C: |\l|>1\}$:}

Let $T = L$ or $R$. This follows from the
Neumann series: $T/\l$ has norm less than one so $T/\l - I$ has a bounded inverse.
Thus $\rho(L), \rho(R) \supset \{\l\in\C: |\l|>1\}$.

To show equality, note that the spectrum of an operator is closed. $\{\l\in\C:
|\l|=1\}$ is in the closure of $\s_r(L) \cup \s_p(R)$, thus it
firstly cannot be in $\rho$ of either operator, and secondly must be the
continuous spectrum of both operators by elimination.

\paragraph{Problem 2}
I show no $\l$ can exist where $A = (T-\l I)$ has a trivial kernel and a
range that is not dense.

If $A$'s kernel is trivial, $\l \notin \{\l_i\}$, else $e_i$ is in the kernel of
$A$. In this case, note further that $A(e_i/(\l_i-\l)) = e_i$, so each $e_i$ is
in the range, and finite linear combinations of $e_i/(\l_i-\l)$ are members of
$\ell^2(\mathbb{N})$, thus because finite linear combinations of $e_i$ are dense,
the range is dense. Thus no lambda exists in $\sigma_r(T)$.

I now show $\lambda \in \overline{\{\l_i\}} \setminus \{\l_i\} \in \sigma_c(T)$.
Let $\l$ be a member of the set. Because $\l$ isn't an eigenvalue, by the same
argument as in the previous paragraph $A = T-\l I$ is injective and the range
is dense. To show that the range isn't closed, let $\l_n\to \l$, with $|\l_n-\l|
< 1/n^2$. Then let $v$ be the vector with $1/n$ in the component corresponding
to $\l_n$ in $T$ and 0 otherwise. The formal preimage of $v$ is the sequence with
$1/(n\l_n-n\l)$ in the positions corresponding to each $\l_n \in T$, but as
$|\l_n-\l| = 1/n^2$, the preimage of $v$ isn't in $\ell^2(\mathbb{N})$ as it
isn't square-summable. Thus $\l \in \sigma_c(T)$.

\paragraph{Problem 3}
All such operators map $\ell^2(\mathbb{N})$ to itself.

a) 0 is a compact operator, and $0 - 0I$ has a nontrivial kernel so 0 is in the
point spectrum.

b) $T = T - 0I$ 
 mapping $e_i \to e_i/k$ has all coordinate vectors in its
range and thus has a dense range, but $(1,1/2,1/3,\ldots)$ has the formal
preimage $(1,1,1,\ldots)$, and thus the range isn't closed. It is compact
because it is the norm limit of finite rank operators sending $e_k\to e_k/k$
for $k < n$.

c) $T = T- 0I$ sending $e_k \to e_{2k}/k$ is compact by the same
argument as the previous, has no kernel, but also has a range that isn't dense
as, for example, $e_1$ cannot be approximated by anything in its range.

\paragraph{Problem 4}
Let $\l \in \s_p(T)$ but $\notin \s_p(T^*)$, specifically, let $Tv = \l v$. Then
\begin{displaymath}
  T^*x^*(v) = x^*(Tv) = \lambda x^*(v).
\end{displaymath}
I now show that the range of $T-\l I$ cannot be dense. Let $v^*$ be the vector
of norm 1 such that $v^*(v) = \|v\|$, which is possible by Hahn-Banach. Suppose
the range is dense, and let $x_n^*$ be a sequence of vectors such that
$(T-\l I)x_n^*\to v^*$. Consider then
\begin{displaymath}
  (T-\l I)x_n^*(v) = Tx_n^*(v)-\l x_n^*(v) = \l x_n^*(v)-\l x_n^*(v) = 0,
\end{displaymath}
which means $x^*_n$ doesn't converge weak*ly to $v^*$, and hence also not in the
norm topology, a contradiction. Thus $\l \in \s_r(T^*)$.

\end{document}