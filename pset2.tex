\documentclass[12pt]{article}
\newcommand{\F}{\mathcal{F}}
\renewcommand{\dim}{\text{dim}}
\renewcommand{\chi}{\mathcal{X}}
\renewcommand{\null}{\mathcal{N}}
\newcommand{\ran}{\mathcal{R}}
\newcommand{\R}{\mathcal{R}}
\renewcommand{\H}{\mathcal{H}}
\newcommand{\K}{\mathcal{K}}
\newcommand{\intr}{\int_{-\infty}^{\infty}}
\usepackage{microtype}
\usepackage{enumitem}
\usepackage{amsmath}
\usepackage{amssymb}
\usepackage[margin=1.25in]{geometry}
\usepackage{enumitem}


\begin{document}
\begin{center}
  Lewis Ho\\
  Functional Analysis\\
  Problem Set 2
\end{center}

\paragraph{Problem 1}
Let $M = \sup\{|T(f,f)|,\ \|f\| = 1\}$. Clearly $M \leq \|T\|$. To show $M \geq
\|T\|$, consider the polarization identity:
\begin{displaymath}
  (Tf, g) = \frac{1}{4}[(T(f+g),f+g)-(T(f-g),f-g)+i(T(f+ig),f+ig)-i(T(f-ig),
  f-ig)].
\end{displaymath}
Because $(Th,h) = (h,Th) = \overline{(Tf,f)}$, $(Tf,f)$ is real, and thus:
\begin{displaymath}
  \text{Re}(Tf,g) = \frac{1}{4}[(T(f+g),f+g)- (T(f-g),f-g)].
\end{displaymath}
By definition, $|(Th,h)|\leq M\|h\|^2$, so:
\begin{align*}
  \text{Re}(Tf,g) &\leq \frac{M}{4}(\|f+g\|^2 - \|f-g\|^2)\\
  & \leq \frac{M}{2}(\|f\|^2 + \|g\|^2),
\end{align*}
by the parallelogram law. Letting $\|f\| = \|g\| = 1$, we see $|\text{Re}(Tf,g)
| \leq M$. Finally, substituting $e^{i\theta}g$ for $g$, we see that from the fact
that 1) for some $\theta$, $\text{Im}(T(f, e^{i\theta}g)) = 0$, and 2) $|T(f, e^
{i\theta}g)|$ is invariant under $\theta$, we conclude that $(Tf, g) \leq M$ for
$f$, $g$ with norms less than 1, and thus $M = \|T\|$.

\paragraph{Problem 3}
\begin{enumerate}[label=\alph*)]
\item The linearity of $T$ follows from the linearity of multiplication. The
  boundedness of $T$ can be shown by noting that:
  \begin{displaymath}
    \|Tf\|^2 = \int_0^1|tf(t)|^2dt \leq \int_0^1|f(t)|^2dt = \|f\|^2.
  \end{displaymath}
  Compactness: take a bounded orthonormal basis spanning $\mathbb{L}^2([0.5,1])$,
  (e.g. the fourier basis) say
  $\{e_i\}$ (define them as 0 elsewhere). Consider the sequence $f_n = e_i/t$:
  it is bounded as
  \begin{displaymath}
    \|f_n\|^2 = \int_{0.5}^1|e_i/t|^2dt\leq \int_{0.5}^1\frac{M}{t^2}dt = M,
  \end{displaymath}
  but $Tf_n = e_n$, which, being orthonormal, contain no convergent subsequence.
\item Suppose $\lambda$ is an eigenvalue for some $f$: $tf(t) = \lambda f(t)$,
  thus $(t-\lambda)f(t) = 0\ \forall t \in [0,1]$. Clearly ths is possible only
  when $f(t) = 0$, i.e. there are no eigenvectors.
\end{enumerate}

\paragraph{Problem 4}

Boundedess: by Pythagoras,
\begin{displaymath}
  \|Tf\|^2 = \|\sum_k\alpha_k\frac{e_{k+1}}{k}\|^2 = \sum_k\frac{\alpha^2_k}{k^2}
  \leq \sum_k \alpha^2_k = \|f\|^2.
\end{displaymath}

Compactness: let $a_n=\sum_k\alpha_ke_k$ have norm $\leq 1$. We can write
\begin{displaymath}
  Ta_n = \sum_{k=1}^\infty\frac{\alpha_ke_{k+1}}{k} = 
  \sum_{k=1}^N\frac{\alpha_ke_{k+1}}{k} + \sum_{k=N+1}^\infty \frac{\alpha_ke_{k+1}}
  {k}
\end{displaymath}
The second term converges to zero in norm as $N \to \infty$, so for any $m$, we can
choose $N$ such that this term is less than $1/{10m}$, and then because the
first term is finite dimensional, there exists a subsequence that converges
in that term, and we can choose some $n_i$ such that the distance
between the first $N$ terms of any two $a_{n_j}$ with $j\geq i$ is also less than
$1/{10m}$. Repeat, this time with the $N$-convergent subsequence, and index
the resultant (sub)subsequence $\{a_m\}$. Clearly for $x, y \geq m$, $\|a_x
- a_y\| \leq \frac{1}{m} \to 0$.

No eigenvectors: suppose $\sum a_ke_k$ was an eigenvector, then there exists
some nonzero coefficient $a_k$. Because $Tf = \sum \frac{\lambda \alpha_{k}{k}
  e_{k+1}}{k}$, $\frac{\lambda\alpha_{k-1}}{k-1} = a_k$, i.e. $a_{k-1}$ is
nonzero and by induction $a_1$ is nonzero. But the coefficient of $e_1$ in $Tf$
is 0, so no eigenvectors can exist.



\paragraph{Problem 5}

Suppose $\lambda_k \to 0$: we can show compactness by the same argument as in
the previous problem. Write:
\begin{displaymath}
  Tf_k = \sum_{k=1}^N\lambda_k\alpha_ke_k + \sum_{k=N+1}^\infty\lambda_k\alpha_ke_k,
\end{displaymath}
and again pick some $f_m$ from nested $N$-convergent subsequences. Note that
the second term decreases in norm to zero exactly because $\lambda_n\to 0$, and
the first sits in a compact set as before.

Conversely, suppose $\lambda_k$ doesn't converge to zero, i.e. $\exists
\varepsilon$ such that for all $N$ there exists $k \geq N$ such that $\lambda_k
> \varepsilon$. Create from this a sequence $K_N$. Clearly $\{e_{K_N}\}$ have
norm one but the image of $\{\lambda_{K_N}e_{K_N}\}$ has no convergent
subsequence
as they are all orthogonal with norm $> \varepsilon$, i.e. are always at least
$\sqrt{2}\varepsilon$ apart, by Pythagoras.

\paragraph{Problem 6}

\paragraph{Problem 7}
\begin{enumerate}[label=\alph*)]
\item An operator is Hilbert-Schmidt if $\sum\|Te_i\|^2 < \infty$, where $e_i$
  is an orthonormal basis. $\{\phi_n\}$ constitutes one such basis, and
  $\|T\phi_n\|^2 = |\lambda_n|$. Thus $T$ is Hilbert-Schmidt only if
  $\sum_k|\lambda_n|^2 < \infty$.
\item Given $\{\phi_n\}$ is a basis for $\mathbb{L}^2([0,1])$, we can write
  $K(x,y) = \sum_i\sum_ja_{ij}\phi_i(x)\phi_j(y)$. Then we find the values of
  $a_{ij}$ satisfying $T\phi_n = \lambda_n$:
  \begin{displaymath}
    \int\sum_i\sum_ja_{ij}\phi_i(x)\phi_j(y)\phi_n(y)dy
    = \int\sum_ia_{in}\phi_i(x),
  \end{displaymath}
  because $\phi_j$ and $\phi_n$ are orthogonal when $j\neq n$. Because $\phi_i$
  are linearly independent, the only way this can equal $\lambda_n\phi_n$ is if
  $a_{jn} =\lambda_n$ when $j=n$ and 0 otherwise. I.e. $K(x,y) = \sum_k
  \lambda_k\phi_k(x)\phi_k(y)$.
\item This is basically the same as part a): an operator is Hilbert-Schmidt if
  and only if $\sum_k\|Te_k\|^2 < \infty$, and $\|Te_k\|^2 = |\lambda_k|^2$ if
  we let $\{e_k\}$ be the set of eigenvectors of our symmetric and compact
  operator. Thus $\sum_k|\lambda_k|^2 < \infty$ is necessary and sufficient.
  

\end{enumerate}

\paragraph{Problem 8}
\begin{enumerate}[label=\alph*)]
\item Let $\{e_k\}$ be a basis of eigenvectors for $T_1$. Then $T_1T_2e_k =
  T_2T_1e_k = T_2\lambda_ke_k = \lambda_kT_2e_k$. In other words, $T_2e_k$ is
  an eigenvector for $T_1$ with eigenvalue $\lambda_k$. Next, to show that
  $e_k$ is an eigenvector of $T_2$ (with possibly eigenvalue 0), we consider
  the alternative: that there is a set of eigenvectors $\{e_i\}$ of $T_1$ which
  all have the same eigenvalues, and $T_2$ takes them to each other. Clearly
  they have to be a closed cycle (i.e. there is some $k$ such that $T_2^ke_k =
  \lambda^ke_k$), else $T_2$ is clearly not compact. However, this would imply
  there was some $k\geq 1$ such that $\sum\mu_i^ka_if_i = C\sum\mu_ia_if_i$,
  where $\sum a_if_i = e_k$ and $f_i$ are eigenvectors of $T_2$. This is
  clearly impossible, unless $\mu_i$ = 0 or 1 (in which case $e_k$ would be an
  eigenvector), so the eigenvectors of $T_1$ must also be those for
  $T_2$, and vice versa.
\item 
\end{enumerate}

\paragraph{Problem 9}

\begin{enumerate}[label=\alph*)]
\item Because $RK = I-A$, we know that $\dim(\null(RK)) \leq \infty$. Clearly
  $\dim(\null(K)) \leq \dim(\null(RK)) \leq \infty$. Then suppose $\dim(\null(R))
  = \infty$, then $\dim(K^{-1}(\null(R))) = \infty$, but that is a subset of
  $\null(RK)$, which we know to be finite dimensional. Thus both are of finite
  dimension.
\item If $f \in \null(K^*)^\bot$,  $f \in \ran(K)$ because $K$ is compact, thus
  $K\phi = f$ has solutions, as does $RK\phi = Rf$. 
\item Define $R^{-1}: \mathcal{H} \to \R(K)$ and $K^{-1}: \R(K)\to\mathcal{H}$
  to be the right- and left-inverses of $R$ and $K$ in the following sense:
  $R^{-1}$ maps any element in $\H$ to its preimage in $\R(K)$, and $K^{-1}$
  maps any element in the range of $K$ to its preimage in $\H$. These functions
  are well defined because in order for $RK$ to be bijective, $R$ must be
  injective and surjective on $\R(K)$, and $K$ injective on $\H$. Clearly $S =
  K^{-1}R^{-1}$ is well defined on $\H$ and is the inverse to $RK$. In this case,
  $KSR = KK^{-1}R^{-1}R = I$, and thus $\R(I-KSR) = \{0\}$.
  This is trivially a subset of $\null(K)$. Likewise the adjoint of 0 is 0,
  so this is true for the adjoint case too. Finally, if $f\in\null(K^*)^\bot$, it
  is in the domain of $SR$ as defined above ($SR = K^{-1}R^{-1}R = K^{-1}$, so
  its domain is $\R(K)$), and clearly $KSRf = KK^{-1}f = f$.
\end{enumerate}


\end{document}