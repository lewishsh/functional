\documentclass[12pt]{article}
\usepackage{microtype}
\usepackage{enumitem}
\usepackage{amsmath}
\usepackage{amssymb}
\usepackage[margin=1.25in]{geometry}
\usepackage{enumitem}
\newcommand{\N}{\mathbb{N}}

\begin{document}
\begin{center}
  Lewis Ho\\
  Functional Analysis\\
  Practice Pset 4
\end{center}

\paragraph{Problem 1}

This is an application of a theorem we proved in class. I proceed with the same
proof. Let $B$ be the closure of the ball in $c_0(\mathbb{N}$ with respect to
the weak* topology. Suppose the statement is false: then there exists some
$x_0 \in \ell^\infty(\mathbb{N})\setminus B$. Because $B$ is convex and closed,
by Hahn-Banach there exists some $\ell \in (\ell^\infty,wk^*)^*$ such that
$|\ell(x)| \leq \alpha < \alpha + \varepsilon \leq |\ell(x_0)|$, for all $x$
in $B$. Note further that generally, $(X^{**},wk^*)^* = X^*$ (WHY?). Let $\ell$
be norm 1. Because $sup(\ell(x)) = 1$, $\ell(x_0) = x_0(\ell) > 1$, violating
our assumption that $\|x_0\| = 1$.

\paragraph{Problem 2}

$x^*_n$ converging weak* to some $x^*$ is equivalent to saying $x^*(x)\to x^*(x)$
for all $x$, meaning that by the uniform boundedness principle $x^*_n$ are norm
bounded.

Further, note that $X$ is isomorphic to a subset of $X^{**}$, and $X^*$ is Banach
if $X$ is. Thus any weakly convergent sequence in $X$ is represented by a
weak*ly convergent sequence in $X^{**}$, and is hence norm bounded.

\paragraph{Problem 3}

Suppose $x_i\to x$ weakly. This in particular means that for any $x^*$, for
every $\varepsilon$ $x_i$ is eventually in the neighborhood $\{y\in X: |(y-x,
x^*)| < \varepsilon\}$, as these are neighborhoods of $x$ in the weak topology.
Thus $(x_i,x^*) \to (x,x^*)$.

Conversely, suppose $(x_i,x^*)\to (x,x^*)$ for all $x^*$. Then given any
weak neighborhood consisted of $\{(x^*_i,\varepsilon_i)\}$, because there
are finite tuples per neighborhood, we simply choose $N$ such that $(x_i, x^*_j)
< \varepsilon_j$ for all $j>N$. The arguments are exactly the same for
weak* convergence.



\end{document}