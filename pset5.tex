\documentclass[12pt]{article}
\newcommand{\R}{\mathbb{R}}
\renewcommand{\H}{\mathcal{H}}
\newcommand{\X}{\mathcal{X}}
\usepackage{microtype}
\usepackage{amsmath}
\usepackage{amssymb}
\usepackage[margin=1.25in]{geometry}
\usepackage{enumitem}
\begin{document}
\begin{center}
  Lewis Ho\\
  Functional Analysis\\
  Problem Set 5
\end{center}

\paragraph{Problem 1}

\begin{enumerate}[label=\alph*)]
\item Consider $A = f^{-1}(U) \in \X^2$, where $f$ is the addition function.
  Because $\X$ is a TVS, $A$ is open in the product topology. Further, because
  0 is in $U$, 0 (or really, (0,0)) is in $A$, and thus some neighborhood
  containing 0, say $X\times Y$, is in $A$, where $X$ and $Y$ are open sets
  of $\X$. Thus $V = X\cap Y$ is open, is a member of $\mathcal{U}$,
  $V\times V \subset X\times Y \subset A$, and $V+V \subset U$.
\item Because of the joint continuity of scalar multiplication, there exists
  in the preimage of $U$ under scalar multiplication a neighborhood of 0,
  $[-c, c]\times W$, where $W$ is an open set. Let $V = W/c$, hence $\alpha V
  \subset U$ for all $|\alpha| \leq 1$. Then define
  \begin{displaymath}
    \mathcal{V} = \bigcup_{|\alpha| \leq 1}\alpha V
  \end{displaymath}
  For any vector $v \in \mathcal{V}$ and $|\beta| \leq 1$, $\beta v \in
  \mathcal{V}$, as $v \in \alpha V$ for some $\alpha$, and then $|\alpha\beta|
  \leq 1$ so $\beta v \in \alpha\beta V \subset \mathcal{V}$. Our set
  thus is open (union of open sets), contains 0, and is balanced.
\end{enumerate}

\paragraph{Problem 2}

Continuity of addition: let $U$ be an open set. I show that for every $x_0 +
y_0 \in U$, there is some open neighborhood around $(x_0,y_0)$ that maps into
$U$.

Because $U$ is open, it contains some neighborhood around $x_0+y_0$:
\begin{displaymath}
  V_{x_0+y_0} = \bigcap_{j=1}^n\{x\in \X\ |\ p_j(x-x_0-y_0) < \varepsilon_j\}.
\end{displaymath}
Consider the neighborhood around $(x_0,y_0)$, $V_{x_0}\times V_{y_0}$,
where $V$ is defined as it is above for ${x_0+y_0}$, except with $\varepsilon_j
/2$ instead of $\varepsilon$. For any $(x,y)$ in our neighborhood:
\begin{displaymath}
  p_j(x+y-x_0-y_0) \leq p_j(x-x_0)+p_j(y-y_0) < \varepsilon_j,
\end{displaymath}
meaning $V_{x_0}\times V_{y_0}$ maps into the neighborhood around $(x_0,y_0)$,
and thus into $U$. Thus addition is continuous.

Continuity of scalar multiplication. Again let $U$ be an open set; I show every
$(\alpha, x_0)$ with $\alpha x_0\in U$ has a neighborhood also mapping into $U$.

Let $V_{\alpha x_0}$ be an open neighborhood contained by $U$:
\begin{displaymath}
  V_{\alpha x_0} = \bigcap_{j=1}^n \{x\in\X\ |\ p_j(x-\alpha x_0) < \varepsilon_j\}.
\end{displaymath}
Then let us construct a neighborhood of $(\alpha, x_0)$. Define
\begin{displaymath}
  W_{x_0} = \bigcap_{j=1}^n\{x\ |\ p_j(x-x_0) < \varepsilon_j/|2\alpha|\},\ 
  c = \inf_{x\in W_{x_0},\ j = 1,\ldots ,n} \frac{\varepsilon_j}{2|p_j(x)|},
\end{displaymath}
and finally our neighborhood $V_{(\alpha, x_0)}$:
\begin{displaymath}
  V_{(\alpha, x_0)} = (\alpha - c, \alpha + c)\times W_{x_0}.
\end{displaymath}

Note that $c$ is nonzero because $n$ is finite, $|p_j(x)|$ is bounded by
$|p(x_0)| + \varepsilon/2\alpha$. Given some $(\beta, x)$ in our neighborhood,
\begin{align*}
  p_j(\alpha x_0 - \beta x)
  & = p_j(\alpha(x_0-x) + (\alpha - \beta)x)\\
  & \leq |\alpha|p_j(x-x_0) + |\alpha -\beta|p_j(x)\\
  & < \varepsilon_j.
\end{align*}
Thus scalar multiplication maps our neighborhood into $V_{\alpha x_0}\subset U$.

\paragraph{Problem 3}

Let $U$ be an open subset of $\X$ and $x_0+y_0 \in U$. By the continuity of
addition, there exists some neighborhood $V_{x_0}\times V_{y_0}$ such that
$(x,y)\in V_{x_0}\times V_{y_0}$ satisfies $x + y \in U$. By the definition of
neighborhoods in a product space, $V_{y_0}$ is a open neighborhood of $y_0$
in $\X$. Thus $V_{y_0} + x_0 \in U$ implies our function is continuous.
The continuity of the inverse is established by repeating the argument with
$-x_0$ instead of $x_0$.

Let $U$ be an open set in $\X$ and let $\alpha_0 x_0 \in U$. By continuity, there
is some open neighborhood $V_{\alpha_0}\times V_{x_0}$ such that $(\alpha,x)$ in
it satisfies $\alpha x \in U$. As above, $V_{x_0}$ is open in $\X$ and satisfies
$\alpha_0V_{x_0} \in U$, thus establishing continuity. The continuity of its
inverse follows with the same argument, replacing $1/\alpha$ for $\alpha$.

\paragraph{Problem 4}

We've shown in class that $x^*$ is continuous in the weak topology. It remains
to be shown that weaker topologies are insufficient.

Let $\tau$ be a strict subset of $wk$, i.e there is some open set in $wk$ that
is not in $\tau$. This implies there's some $x$ in the set with a
neighborhood in $wk$ but not $\tau$, i.e.
\begin{displaymath}
  V_{x_0} = \bigcap_{i=0}^n\{x\in\X\ |\ |\ell_i(x-x_0)|<\varepsilon_i\}\notin\tau 
\end{displaymath}
This in turn implies for one of the $\ell_i$, its $\varepsilon_i$ cylinder is
not an open set of $\tau$, (otherwise $V_{x_0}$, a finite intersection of all
such cylinders, would too be open). The $\varepsilon_i$ cylinder can be
rewritten as $\ell_i^{-1}((\ell(x_0)-\varepsilon_i, \ell(x_0)+\varepsilon_i))$,
i.e. $\ell_i\in \X^*$ is not continuous in $\tau$ as our interval is an open
set in our field.

\paragraph{Problem 5}

We've shown in class that $(\X^*,wk^*)^* = \X$. Let $\tau$ be a yet weaker
topology; thus there is an open set and within it a neighborhood of some
$\ell_0^*$ that is not in $\tau$, i.e.
\begin{displaymath}
  V_{\ell_0^*} = \bigcap_{i=0}^n\{\ell\in\X^*\ |\ |(\ell-\ell^*)(x_i)| <\varepsilon_i
  \} \notin\tau.
\end{displaymath}
This means that some $\varepsilon_i$ cylinder of some $\ell_i$ isn't an open
set in $\tau$. Note that:
\begin{displaymath}
  \{\ell\in\X^*\ |\ |(\ell-\ell^*)(x_i)|<\varepsilon_i\}
  = x_i^{-1}((\ell^*(x_i)-\varepsilon_i,\ell^*(x_i)+\varepsilon_i))\notin\tau.
\end{displaymath}
Thus $x_i$ isn't continuous in $\tau$.

\paragraph{Problem 6}

With some algebra:
\begin{align*}
  \|h_n-h\|^2
  &= \sum_i((h_n)_i-h_i)^2\\
  &= \sum_i(h_n)_i^2 - 2\sum_i(h_n)_ih_i + \sum_ih_i^2\\
  &= \sum_i(h_n)_i^2 - \sum_ih_i^2 + 2\sum_ih_i^2 - 2\sum_i(h_n)_ih_i\\
  &= (\|h_n\|^2 - \|h\|^2) + 2((h,h) - (h_n,h)).
\end{align*}
The first converges to zero by the convergence \emph{of} norms, and the second by
weak convergence.

\paragraph{Problem 7}

Let $\bar{S}$ denote the weak closure of $S$. Let $R = \{x:\|x\| \leq 1\}$.

$\bar{S}\subset R$: to show that every element in the closure of $S$ must have
norm less than one, suppose it isn't true: there is some $x_i \to x$ where
$x_i$ are norm 1 but $\|x\| > 1$. By Hahn-Banach, however, there is some linear
functional $\ell$ for which for all $y \in \overline{B_1(0)}$, $|\ell(y)-\ell(x)
| > \varepsilon$. Note that $x_i$ in particular are members of $\overline{B_1(0)}
$, thus they never enter the $\varepsilon$ neighborhood of $x$ induced by $\ell$,
contradicting our assumption that $x_i\to x$ weakly.

$R\subset \bar{S}$:

% want to show: we can construct a sequence st for each neighborhood of x
% x_i is eventually in it.

\end{document}