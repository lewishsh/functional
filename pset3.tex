\documentclass[12pt]{article}
\newcommand{\R}{\mathbb{R}}
\renewcommand{\H}{\mathcal{H}}
\newcommand{\N}{\mathbb{N}}
\usepackage{microtype}
\usepackage{enumitem}
\usepackage{amsmath}
\usepackage{amssymb}
\usepackage[margin=1.25in]{geometry}
\usepackage{enumitem}
\begin{document}
\begin{center}
  Lewis Ho\\
  Functional Analysis\\
  Problem Set 3
\end{center}

\paragraph{Problem 1}

Any linear functional can be fully determined by its actions on a basis by
linearity:
\begin{displaymath}
  f\left(\sum_ka_ke_k\right) = \sum_k a_kf(e_k) \in \mathbb{R}
\end{displaymath}
thus any $f$ can be defined as a sequence $\{a_n\}$ where $f(e_k) = a_n$. If
$\{a_n\} \in \ell^1(\mathbb{N})$, for any $\{b_n\} \in c_0(\N)$, $|f_{a}
(\sum_kb_k)| = |\sum_ka_kb_k| \leq \sup(|b_n|)\sum|a_k|  = M\|\{b_n\}\|$. 
Note that we assume the norm defined on $c_0(\N)$ is the supremum norm, which
is well defined because convergent sequences are bounded.
 Hence $\ell^1(\N) \subseteq
c_0(\N)^*$.

Conversely, given some $\{a_n\}$ such that $\sum|a_n| = \infty$, we can find a
sequence of indices $n_1,\ n_2,\ldots$ such that $\sum_{i=1}^{n_k}|a_i| \geq
k^2$. Consider then the sequence $\{b_n\}$, with $b_i = |1/k|$ for $n_{k-1} <
i \leq n_k$, and the sign of $b_i$ the same as the sign of $a_i$. Then:
\begin{displaymath}
  f_a(\{b_n\}) \geq \sum_{i=1}^{n_k}a_ib_i \geq \frac{1}{k}\sum_{i=1}^{n_k}|a_i|
  \geq k^2/k \to \infty.
\end{displaymath}
Note that $\|\{b_n\}\| \leq 1$.
Thus $f$ is unbounded and $\ell^1(\N) \supseteq c_0(\N)^*$.

\paragraph{Problem 2}

Suppose not. Let $X=\ell^\infty(\N)$ contain a countable subset $\{w_i\}$ that is
dense. With Gram-Schmidt we obtain a orthonormal basis $\{v_i\}$ that is
countable. Consider the map $U(v_i) = e_i$, where $\{e_i\}$ is the standard
``basis''---sequences of zeros, except for 1 in the $i$th element. For any
$f = \sum_ia_iv_i$,
\begin{displaymath}
  \|U(f)\| = \|U(\sum_ia_iv_i)\| = \|\sum_ia_iU(v_i)\|
  = \|\sum_ia_ie_i\| = \sup_i\{a_i\} = \|f\|
\end{displaymath}
$U$ is clearly linear and bijective, and thus
$X$ is unitarily equivalent to the closure of the space spanned
by $\{e_i\}$ with the same norm, say, $Y$.

But $Y$ is not separable. Let $\{a_n\}$ be the sequence with $a_n = 1$ for
all $n$. For any finite linear combination $\sum_i\alpha_ie_{k_i}$ of $\{e_i\}$,
\begin{displaymath}
  \|\{a_n\} - \sum_i^N\alpha_ie_{k_i}\|_\infty \geq 1
\end{displaymath}
as $a_n$ has infinite terms that are 1 compared to the finite ones in finite
linear combinations of $\{e_i\}$. Hence there is always some $j$ where the
$j$th term of $\{a_n\}$ is 1 and that of our linear combination is 0.

The fact that $X$ is also cannot be separable follows from unitary equivalence.
Suppose not: consider $U^{-1}(\{a_i\}) = b$. By separability, $\sum_i^nc_iv_i
\to b$ as $n\to \infty$, i.e. for any given $\varepsilon$, we can find some
$N$ such that for $n\geq N$:
\begin{displaymath}
  \|b - \sum_i^nc_iv_i\| < \varepsilon.
\end{displaymath}
By the properties of $U$,
\begin{displaymath}
  \|U(b - \sum_i^nc_iv_i)\| = \|U(b)- U(\sum_i^nc_iv_i)\| =
  \|\{a_n\} - \sum_i^Nc_ie_i\| < \varepsilon
\end{displaymath}
in which case linear combinations of $\{e_i\}$ converge to $\{a_i\}$, a
contradiction. Thus we've shown that no subset of $X$ can be dense.



\paragraph{Problem 3}

It is a subspace by the fact that if $a_n \to a$ and $b_n \to b$, $\lim_{n\to
  \infty} a_n + b_n $ exists, and is in fact $a + b$ (this all follows from the
continuity of addition $\mathbb{R}^2 \to \mathbb{R}$). Thus $\{a_n\} + \{b_n\}
\in c$. Closure: let $\{b_n\}$
be the limit of a sequence of sequences $\{a_n\}_k$. For any $\varepsilon > 0$,
there exists some $c_n = \{a_n\}_i$ such that $\sup|c_n - b_n| < \varepsilon/3$,
and also some $N$, because $c_n$ is convergent, such that $|a_m - a_n| <
\varepsilon/3$ for all $m,\ n > N$. Thus for any $m,\ n > N$,
\begin{displaymath}
  |b_n - b_m| \leq |b_n-c_n| + |b_m-c_m| + |c_n-c_m| < \varepsilon,
\end{displaymath}
i.e. $\{b_n\}$ is Cauchy and thus in $c$ as well. $c$ is therefore closed.

\paragraph{Problem 5}
$\ell(f) = f(x_0) \leq \sup(f) = \|f\|$ on $[0,1]$. Additionally, there are
continuous functions that attain their maximum on $x_0$, thus the
inequality is sometimes an equality and $\|\ell\| = 1$.

\paragraph{Problem 6}
\begin{enumerate}[label=(\alph*)]
\item 
  Let $f,\ g \in E_\alpha$. For all $\lambda \in \mathbb{R}$, $\lambda f(0) +
  (1-\lambda)g(0) = \lambda \alpha + (1-\lambda) \alpha = \alpha$, thus all
  convex combinations of $f$ and $g$ are clearly also in $E_\alpha$.

  Then, given
  any $f \in X$ and $\varepsilon > 0$, there exists some continuous function
  $g \in X$ such that $\|f - g\| < \varepsilon/2$, (given the density of the
  continuous functions in $L^2$). Then we can find another function $h \in E_\alpha
  $ such that $\|g-h\| < \varepsilon/2$, where $h = g$ in $[-1,1]\setminus [-\delta
  ,\delta]$, and $h$ decreases/increases linearly to $\alpha$ at $0$. We can find
  such a close $h$ because we can, by decreasing $\delta$, decrease the measure
  of the set on which $h$ differs from $g$ arbitrarily small, and this difference
  (squared) is bounded given the continuity of both functions and the compactness
  of our domain. Thus $\|f-h\| < \varepsilon$, i.e. $E_\alpha$ is dense.
\item No function can attain the value of both $\alpha$ and $\beta$ at 0. Further,
  because all functions we are considering are continuous, functions in $E_\alpha$
  and $E_\beta$ must differ in an area of nonzero measure, thus cannot belong to
  the same equivalence class in $L^2$. Thus both sets are disjoint. To show they
  cannot be separated, we first prove a lemma.

  Lemma: if $A$ is a dense set, there exist no nonzero bounded linear functionals
  such that $\ell(a) \leq C$. Proof: because $\ell$ is nonzero, there exists some
  $a$ such that $\ell(a) \neq 0$. By linearity, we can scale $a$ to find an element
  $b$ such that $\ell(b) > C + \varepsilon$. Because $A$ is dense, we can find some
  $f\in A$ such that $\|f - b\| < \varepsilon/M$, where $M$ is the norm of $\ell$.
  This means $\|\ell(f) - \ell(b)\| < \varepsilon$, and thus $\ell(f) > C$.

  From the lemma it follows that because $E_\alpha$ is dense, $\{\ell(E_\alpha)\}$
  has no supremum, and thus no such separation is possible. Geometric Hahn Banach
  is not possible because the sets are not compact. Note that we assume here that the
  question refers to nonzero functionals, because as far as I can tell such an
  arrangement is true for $\ell = 0$.
  
\end{enumerate}

\paragraph{Problem 7}

Disjointness: a polynomial cannot both have a negative leading
coefficient and all non-negative coefficients, thus they are disjoint.
Convexity: the sum of two polynomials with negative leading coefficients has
a negative leading coefficient, this being either the sum of both or just one
of them; and the sum of two polynomials with non-negative coefficients has
non-negative coefficients as well. Finally, multiplying polynomials with
$0 < \lambda \leq 1$ doesn't change the sign of its coefficients, thus both
$A$ and $B$ are convex.

Suppose there does exist some nonzero $\ell$ such that $\ell(a) \leq \ell(b)$,
for all $a$ in A and $b$ in $B$,
then by the completeness of $\mathbb{R}$, there exists a real $C$ (e.g. the
supremum of $\ell(a)$) such that $\ell(a)\leq C\leq \ell(b)$ for all $a$ and
$b$. Because $0 \in B$, by the linearity of $\ell$, $C \leq \ell(0) = 0$.
However, because for any monomial $a \in A$, $a/n \in A$ also, for $n>0$, thus
$\ell(a/n) = \ell(a)/n \to 0$ as $n\to\infty$. Thus $C \geq 0$, i.e. $C = 0$.

If $\ell$ is nonzero, there must be some monomial $x^n$ such that $\ell(x^n) 
\neq 0$,
as all polynomials are finite sums of monomials, and thus if they all mapped
to 0, $\ell$ would be 0 on all of $\mathcal{P}$. Clearly the sign of $\ell(ax^n
)$ is the same as the sign of $a$. Consider then the polynomial $p = -x^{n+1}
+ax^n \in A$: because we can make $a$ and thus $\ell(ax^n)$
arbitrarily large, $\ell(-x^{n+1})$ cannot be finite and yet keep $\ell(p) = 
\ell(-x^{n+1}) + \ell(ax^n)$ to remain $\leq 0$ for all $a>0$ i.e. no such
linear functional can exist.

\paragraph{Problem 8}
Let $K_1$ be the half space with $y \leq 0$, and let $K_2 = \{(x,y) \in \R^2\
|\ x \geq 1,\ y \geq 1/x\}$. Both sets are convex: the half space is convex,
and $\frac{1}{x}$ is a convex function, and truncating it like so
doesn't hurt this convexity.
Because the boundary of $K_1$ is the $x$ axis,
the only hyperplanes (lines) not intersecting $K_1$ must be parallel to the $x$
axis and $\geq 0$, i.e. $\ell(x,y) = ay$, where $a \geq 0$. However, the infimum
for $y$ where $(x,y) \in K_2$ is 0, thus:
\begin{displaymath}
  \sup_{x\in K_1}\ell(x) = 0 = \inf_{y\in K_2}\ell(y),
\end{displaymath}
with this construction. I.e. no such $\ell$ exists that satisfies our
requirement.

\paragraph{Problem 9}
\begin{enumerate}[label=(\alph*)]
\item 
  Let $a^*\in Z^*$ and $b\in X$. Because $T$ and $S$ are bounded, their adjoints
  $T^*$ and $S^*$ exist and are bounded. Note that for any $A: X\to Y$, $y^*\in
  Y^*$, $x\in X$, $(y^*, Tx)_Y = y^*Tx = T^*y^*x = (T^*y^*,x)_X$ by definition.
  Consider then $(z^*,STx)_Z$:
  \begin{displaymath}
    (z^*,STx)_Z = (S^*z^*,Tx)_Y = (T^*S^*z^*,x)_X,
  \end{displaymath}
  i.e. $T^*S^*z^* = z^*ST$ for all $z^* \in Z^*$, which is the domain of both
  $(ST)^*$ and $T^*S^*$. This last equation is how we define $(ST)^*$, thus $(ST)^*
  z^* = T^*S^*z^*\ \forall z^*$, hence $(ST)^* = T^*S^*$.
\item Again we need to show that $(aS^* + bT^*)y^* = y^*(aS+bT)$ for all $y^*
  $ in $Y^*$, i.e. $(aS^*+bT^*)y^*(x) = y^*(aS+bT)(x)$ for all $y^*\in Y^*$,
  $x\in X$:
  \begin{displaymath}
    (aS^*+bT^*)y^*(x) = aS^*y^*(x)+bT^*y^*(x)
  \end{displaymath}
  then by the definition of $T^*$ and $S^*$:
  \begin{displaymath}
    = ay^*S(x)+by^*T(x) = y^*aS(x)+y^*bT(x) = y^*(aS + bT)(x)
  \end{displaymath}
  by linearity.
\item We show $(T^{-1})^*: X^*\to Y^*$ is the inverse map of $T^*:Y^*\to X^*$:
  \begin{displaymath}
    (T^{-1})^*T^*y^*(a) = (T^{-1})^*y^*T(a) = y^*T^{-1}T(x) = y^*(a)
  \end{displaymath}
  likewise,
  \begin{displaymath}
    T^*(T^{-1})^*x^*(b) = T^*x^*T^{-1}(b) = x^*TT^{-1}(b) = x^*(b)
  \end{displaymath}
  for all $a\in Y,\ b\in X,\ x^*\in X^*\, y^*\in Y^*$, by repeated application
  of the $A^*b^* = b^*A$ rule. Thus $(T^{-1})^* = (T^*)^{-1}$.

\end{enumerate}
\end{document}